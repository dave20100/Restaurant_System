\documentclass{article}

\usepackage[T1]{fontenc} 
\usepackage[utf8]{inputenc}
\usepackage[polish]{babel}




\renewcommand{\contentsname} {Spis treści}

\usepackage{graphicx}

\begin{document}
 
\begin{titlepage} 
	\newcommand{\HRule}{\rule{\linewidth}{0.5mm}} 
	
	\center 
	
	\textsc{\LARGE POLITECHNIKA WROCŁAWSKA WYDZIAŁ ELEKTRONIKI}
	
	\HRule\\[3.0cm]
	
	{\huge Projekt z rozproszonych i obiektowych systemów baz danych}\\[2.0cm] 
	
	{\huge\bfseries Rozproszony system bazodanowy przeznaczony do obsługi oddziałów restauracji}\\[2.0cm] 
	
	
	

	\begin{minipage}{0.4\textwidth}
		\begin{flushleft}
			\large
			\textit{AUTORZY}\\
			Michał Kalinowski\\
			Paweł Kolak\\
			Dawid Mikowski\\
		\end{flushleft}
	\end{minipage}
	~
	\begin{minipage}{0.4\textwidth}
		\begin{flushright}
			\large
			\textit{PROWADZĄCY ZAJĘCIA}\\
			Dr inż. Robert Wójcik
			\\[2.0cm] 
			OCENA PRACY:
		\end{flushright}
	\end{minipage}
	
	% If you don't want a supervisor, uncomment the two lines below and comment the code above
	%{\large\textit{Author}}\\
	%John \textsc{Smith} % Your name
	
	%------------------------------------------------
	%	Date
	%------------------------------------------------
	
	\vfill\vfill\vfill % Position the date 3/4 down the remaining page
	\HRule\\[0.5cm]
	{\large\today} % Date, change the \today to a set date if you want to be precise
	
	%------------------------------------------------
	%	Logo
	%------------------------------------------------
	
	%\vfill\vfill
	%\includegraphics[width=0.2\textwidth]{placeholder.jpg}\\[1cm] % Include a department/university logo - this will require the graphicx package
	 
	%----------------------------------------------------------------------------------------
	
	\vfill % Push the date up 1/4 of the remaining page
	
\end{titlepage}
\clearpage

\tableofcontents
 \newpage
\section{Wstęp}
	\subsection{Cele projektu}

	\subsection{Założenia projektu}

	\subsection{Zakres projektu} 

 
\section{Replikacja w systemie bazy danych ...}
	\subsection{Pojęcie replikacji i podstawowe informacje}

	\subsection{Replikacja transakcyjna}

	\subsection{Replikacja master-slave}

\section{Model konceptualny i dizyczny bazy danych}
	\subsection{Model konceptualny}

	\subsection{Model fizyczny}

\section{Implementacja bazy danych w środowisku ...}
	\subsection{Realizacja bazy danych}

	\subsection{Definiowanie łączników}
	
	\subsection{Wykorzystanie mechanizmów replikacji transakcyjnej}

	\subsection{Wykorzystanie mechanizmów replikacji master-slave}
	
\section{Projekt i implementacja aplikacji klienckiej}
	\subsection{Funkcje aplikacji - diagram przypadków użycia}

	\subsection{Ralizacja wybranych funkcjonalności}
	
\section{Projekt i implementacja aplikacji administratorskiej}
	\subsection{Funkcje aplikacji - diagram przypadków użycia}

	\subsection{Ralizacja wybranych funkcjonalności}
	
\section{Wdrożenie i testowanie aplikacji}

\section{Podsumowanie}

 
1
 
\end{document}